
\documentclass{report}

\include{preamble}
\include{macros}
\include{letterfonts}

\usepackage[Glenn]{fncychap}
\usepackage{mathtools}
\DeclarePairedDelimiter\bra{\langle}{\rvert}
\DeclarePairedDelimiter\ket{\lvert}{\rangle}
\DeclarePairedDelimiterX\braket[2]{\langle}{\rangle}{#1 \delimsize\vert #2}

\title{\Huge{Solutions to Quantum Mechanics: The Theoretical Minimum}}
\author{\huge{}}
\date{}


\begin{document}

\maketitle
\newpage
\pagebreak

\section*{Lecture 1}
\qs{(1.1)}
{
a) Using the axioms for inner products, prove
\begin{equation*}
\{\bra{A} + \bra{B}\}\ket{C} = \braket{A}{C} + \braket{B}{C}.
\end{equation*}
b) Prove $ \braket{A}{A} $ is a real number.
}

\begin{solution}
a)\\
Let $\bra{U} = \bra{A} + \bra{B}$, then $\{\bra{A} + \bra{B}\}\ket{C} = \braket{C}{U}^*$ according to the second \textit{inner-product} axiom. From the first \textit{inner-product} axiom the inner-product can be rewritten to:
\begin{equation*}
\braket{C}{U}^* \iff [\bra{C}\{\ket{A}+\ket{B}\}]^* \iff [\braket{C}{A}+\braket{C}{B}]^* \iff \braket{C}{A}^* + \braket{C}{B}^*
\end{equation*}
By using the second inner-product axiom again, this gives us $\braket{A}{C} + \braket{B}{C}$. \textbf{Q.E.D}!\\\\\\
b)\\
From p.30 we know that $\bra{A}$ and $\ket{A}$ can be represented as a column and a row vector respectively:
\begin{align*}
\bra{A} &=
\begin{pmatrix}
a_1^* a_2^* \cdots a_n^* 
\end{pmatrix} \\
\ket{A} &=
\begin{pmatrix}
a_1 \\
a_2 \\
\vdots \\
a_n 
\end{pmatrix}
\end{align*}
Where $a_i^*$ denotes the complex conjugate of the scalar. Then $\braket{A}{A}$ is a simple "dot product" as demonstrated on p.31:
\begin{equation*}
\braket{A}{A} =
\begin{pmatrix}
a_1^* a_2^* \cdots a_n^* 
\end{pmatrix}
\begin{pmatrix}
a_1 \\
a_2 \\
\vdots \\
a_n 
\end{pmatrix} = a_1^* a_1 + a_2^* a_2 + \cdots + a_n^* a_n
\end{equation*}
Let $a_j = \alpha_j + \beta_j i$, then $a_j^* = \alpha_j - \beta_j i$, where $i = \sqrt{-1}$, $a_j$ is a component of $\ket{A}$, and $\alpha_j,\:\beta_j \in \mathbb{R}$.\\
 It follows that $a_j a_j^* = \alpha_j^2 +\alpha_j \beta_j i - \alpha_j \beta_j i - \beta_j^2 (-1)^2 = \alpha_j^2 + \beta_j^2$. By this we show that a multiplication of a complex scalar with its conjugate gives a \textbf{real} number. Thus the sum, $a_1^* a_1 + a_2^* a_2 + \cdots + a_n^* a_n$ is also a \textbf{real} number. \textbf{Q.E.D}!
\end{solution}

\end{document}
